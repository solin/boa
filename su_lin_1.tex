\title{Study on the finite element calculation method of Multibody system}
\author{} \institute{}
\tocauthor{C.~Su, \underline{L.~Su}}
\maketitle

\begin{center}
{\large Chao Su, \underline{Lin Su}}\\
College of Water Conservancy and Hydropower Engineering, HoHai University\\
{\tt csu\_hhu@163.com, sulin1987@tamu.edu}
\end{center}

\section*{Abstract}
Contact problems of few blocks are a part of hydropower engineering stability problems, such as arch dam abutments, underground caverns, slope stability, etc. Finite element method of continuous problems can't reflect actual contact state, while discrete element method has excessive variables and it's hard to converge. According to the basic idea of displacement method and force method from classical mechanics, every single elastic block can be discretized with isoparametric element. Single block satisfies Centre-of-mass Motion Theorem and Theorem of moment of momentum relative to the Center of mass. Taking interfacial contact force and global motion of blocks as variables, governing equations can be established. Based on the iterative calculation of governing equations, contact state among blocks can be determined, contact force and global displacement of blocks can be computed, as well as the deformation and stress.

\bibliographystyle{plain}
\begin{thebibliography}{10}
\bibitem{Cundall71}
{\sc Cundall P A}. {A computer model for simulating progressive large scale movement in block rock System}. Symposium ISRM. 2 (1971), pp.~129--136.
	
\bibitem{ShiGoodman85}
{\sc Genhua Shi, Goodman Richard E}. {Two dimensional discontinuous deformation analysis}. Int J Numer Anal Methods Geomech. 9 (1985),pp.~541--556.

\bibitem{A1}
{\sc Jun Liu, Jianyun Chen, Xianjing Kong, Gao Lin}.{Seismic stability analysis of rolled compacted concrete dam based on approach of DDA coupled with FEM}. Journal of Dalian University of Technology. 43 (2003), pp.~793--798.

\bibitem{A2}
{\sc Qinghui Jiang, Chuanyun Zhu}.{Discoutinuous deformation analysis for dynamics and its application in modeling of blasting throwing process}. Engineering Blasting. 10 (2004), pp.~5--8.

\bibitem{A3}
{\sc Yong Li, Maotian Luan, Xiating Feng, Yongjia Wang, Xiangji Ye}.{Viscoelastic analysis method of discontinuous deformation computational mechanicsmodel \Rmnum{1}: Fundamental theory}. Chinese Journal of Computational Mechanics. 21 (2004), pp.~440--447.
\end{thebibliography}