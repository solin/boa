\title{Monolithic Modeling of Assembly of Induction Shrink Fits}
\author{} \institute{}
\tocauthor{\underline{P.~Karban}, F.~Mach, I.~Dole\v{z}el}
\maketitle

\begin{center}
{\large \underline{Pavel Karban}, Franti\v{s}ek Mach}\\
Faculty of Electrical Engineering, University of West Bohemia, RICE\\
{\tt karban@kte.zcu.cz, fmach@kte.zcu.cz}\\
\vspace{4mm}
{\large Ivo Dole\v{z}el}\\
Faculty of Electrical Engineering, Czech Technical University\\
{\tt dolezel@fel.cvut.cz}
\end{center}

\section*{Abstract}
Assembly of an induction shrink fit is modeled in monolithic formulation. This process is realized by induction heating of one part (usually that with higher coefficient of thermal dilatability) whose dimensions enlarge due to its thermal dilatation; another cold part is then inserted into the hole of the first part (or, vice versa, the hot part is drawn on the cold part). After consequent cooling of the whole system we obtain the shrink fit. The model of the process represents a triply coupled strongly nonlinear problem (typical by the interaction of magnetic field, temperature field and field of thermoelastic displacements) whose solution must be performed in hard-coupled formulation, otherwise the results may exhibit unacceptable errors.

Modeling of induction shrink fits is still a challenging problem and the relevant references are rather rare. Quasi-coupled formulation of the task can be found in \cite{Skopek} and \cite{Ukraine}, but only with very approximate respecting of nonlinearities. This paper treats the problem much more accurately. The authors describe its complete mathematical model consisting of three nonlinear partial differential equations and illustrate its solution by numerical computation of a typical example. The computations are carried out by codes Hermes2D and Agros2D \cite{Hermes2D} based on the finite element method of higher order of accuracy and algorithms described partly in \cite{Solin}. The solution is supplemented with the analysis of the influence of the principal nonlinearities on the accuracy of results.

This work was supported by the European Regional Development Fund and Ministry of Education, Youth and Sports of the Czech Republic under the project No. CZ.1.05/2.1.00/03.0094: Regional Innovation Centre for Electrical Engineering (RICE) and by Grant project GACR P102/11/0498.

\bibliographystyle{plain}
\begin{thebibliography}{10}

\bibitem{Skopek}
{\sc M.~Skopek, B.~Ulrych, and I.~Dole\v{z}el}. {Optimized regime of induction heating of a disk before its pressing on shaft}. IEEE Trans. on Magn. 37 (2001), pp.~3380--3383.

\bibitem{Ukraine}
{\sc I.~Dole\v{z}el, P.~Karban, B.~Ulrych, M.~Pantelyat, Y.~Matyukhin, P.~Gontarowsky, and N. Shulzhenko}. {Limit operation regimes of actuators working on principle of thermoelasticity}. IEEE Trans. on Magn. 44 (2008). pp.~810--813.

\bibitem{Hermes2D}
{http://hpfem.org}.

\bibitem{Solin}
{\sc P.~Solin, J.~Cerveny, L.~Dubcova, and D.~Andrs}. {Monolithic discretization of linear thermoelasticity problems via adaptive multimesh \textit{hp}-FEM}. J. Comput. Appl. Math. 234 (2010), pp.~2350--2357.
\end{thebibliography}