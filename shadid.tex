

\title{Solution Methods for Multiple-time-scale Multiphysics Systems: Application to Transport/Reaction and Resistive MHD}
\author{} \institute{} % Intentionally left blank
\tocauthor{J. N. Shadid et. al.}
\maketitle
\begin{center}
{\large J. N. Shadid, R. P. Pawlowski, E. C. Cyr, P. T. Lin, R. S. Tuminaro}\\
Sandia National Laboratories\\
{\tt jnshadi@sandia.gov}\\
\vspace{4mm} % Use this space when including 3rd author
{\large L. Chacon}\\
Oak Ridge National Laboratory\\
\end{center}

\section*{Abstract}

A current challenge before the computational science and numerical
mathematics community is the efficient computational solution of
multiphysics systems.  These systems are strongly coupled, highly
nonlinear and characterized by multiple physical phenomena that span a
very large range of length and time scales.  
These characteristics make the scalable, robust, 
accurate, and efficient computational solution of these systems over,
relevant dynamical time scales of interest extremely challenging.

In this presentation I will discuss issues related to the stable,
accurate and efficient time integration, nonlinear, and linear solution of
multiphysics systems. The discussion will begin with a few illustrative
examples that compare operator split and semi-implicit
approaches, to fully-coupled fully-implicit methods. 
I will then overview a number of the important fully-coupled 
solution methods that our research group has applied to 
the solution of coupled multiple-time-scale
multi-physics systems. 
The solution methods that we employ
include, fully-implicit time integration, direct-to-steady-state
solution methods, continuation, bifurcation, and optimization
techniques that are based on Newton-Krylov iterative solvers [1,2].  
To enable the robust, scalable and efficient solution of large-scale sparse 
linear systems algebraic multilevel (AMG) preconditioners are employed. 
These include a fully-coupled graph-based aggregation AMG technique [3] and 
approximate block factorization techniques [4].
To demonstrate the capability of these methods I will present
representative results for solution of transport / reaction and resistive 
magneto-hydrodynamic systems with stabilized finite element methods. 
In this context I will discuss robustness, efficiency, and the parallel and algorithmic
scaling of the solution methods.

*This work was partially supported by  the DOE office of Science AMR program at Sandia National Laboratory. Sandia is a multiprogram laboratory operated by Sandia Corporation, a Lockheed Martin Company, for the United States Department of Energy's National Nuclear Security Administration under contract DEM-AC04-94AL85000

\bibliographystyle{plain}
\begin{thebibliography}{10}

\bibitem{EwingWangYang03}
{\sc M. Sala and R.S. Tuminaro}. {Jacobian-free {N}ewton--{K}rylov methods: a survey of approaches and applications}. J. Comput. Phys 31 (2004), pp.~357--397.

\bibitem{CockburnGopalakrishnan04}
{\sc J. N. Shadid, A. G. Salinger, R. P. Pawlowski, P. T. Lin, G. L. Hennigan, R. S. Tuminaro and R. B. Lehoucq}. {Large-scale Stabilized FE Computational Analysis of Nonlinear Steady State Transport/Reaction Systems}. CMAME 195
  (2006), pp.~1846--1871.

\bibitem{EwingWangYang03}
{\sc M. Sala and R.S. Tuminaro}. { A new Petrov-Galerkin smoothed aggregation preconditioner for nonsymmetric linear systems}. SIAM 
J. Sci. Stat. 31 (2008), pp.~143--166.

\bibitem{A104}
{\sc H. Elman, V. Howle, J. N.  Shadid, R. Shuttleworth, and R. Tuminaro}.
\newblock A Taxonomy of Parallel Mulit-level Block Preconditioners for the Incompressible Navier-Stokes Equations.
\newblock J. Comp. Phys. 227 (2008), pp. 1790 --  1808. 

\end{thebibliography}
