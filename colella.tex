\title{Progress on High-Order Finite-Volume Methods}
\author{} \institute{}
\tocauthor{P. Colella}
\maketitle

\begin{center}

{\large Phillip Colella}\\
Computational Research Division, Lawrence Berkeley National Laboratory \\
{\tt PColella@lbl.gov}

\end{center}

\section*{Abstract}

In this talk, we will discuss recent progress in the design of higher-order (accuracy greater than second-order) finite-volume methods for conservation laws, i.e.  methods based on computing the divergence of a flux exactly, then introducing approximations in the quadrature rules for computing the averages of fluxes over cell faces. Topics to be discussed include: higher-order quadrature methods for computing flux integrals, and the application to time-dependent problems by using method of lines; methods for hyperbolic conservation laws using explicit Runge-Kutta methods, including high-order preserving limiters and adaptive mesh refinement; extension to semi-implicit methods for advection-diffusion problems and incompressible flow; and the extension to mapped and mapped-multiblock methods. We will show the use of these methods using examples from fluid dynamics and plasma physics.