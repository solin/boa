\title{Modeling the Influence of Wind Characteristics and the Atmospheric Parameters on the Wind Turbine Performances}
\author{} \institute{}
\tocauthor{\underline{R.~Belu}, D.~Koracin}
\maketitle

\begin{center}
{\large \underline{Radian Belu}}\\
Drexel University\\
{\tt rbelu@dri.edu}\\
\vspace{4mm}

{\large Darko Koracin}\\
Desert Research Institute\\
{\tt darko@dri.edu}
\end{center}

\section*{Abstract}
The power output uncertainty is closely related to the uncertainty of the wind velocity and other meteorological parameters and their effects on wind turbine performances. For example, there is an inherent uncertainty in the power curve estimate by using the wind speed measured at the hub height. The assumption behind this is that these wind speeds are representative of the wind over the whole rotor area.  While this assumption was adequate for smaller wind turbines this is essentially not true for modern multi-MW ones, and considerable deviations often occur between the expected and produced power. Wind shear and direction, turbulence, and atmospheric stability vary with height as a result of either meteorological and/or terrain conditions. This calls for an adoption of new measurement and power estimation methods (Sumner and Masson, 2006, Wagner et al., 2007). The rotor size combined with the hub height implies that wind turbines are often exposed to highly varying wind conditions with large wind shears, turbulence and atmospheric stability variations within the rotor span. These may affect both the production and the structural safety of the turbine. Consequently, a better power or load prediction requires more representative wind measurements and power computations over the turbine's rotor. The impact of atmospheric stability, turbulence, wind shear and wind gust on the wind turbine output power and site evaluation are reviewed. Velocity, temperature, and turbulence intensity profiles are generated using Monin-Obukov similarity theory. The resulting system of nonlinear equations was solved numerically and tested against field observations. The experimental data used in this study were collected during 14 months on an 80 m instrumented meteorological tower at four levels (10 m, 40 m, 60 m, and 80 m), using sonic anemometer. The rotor's averaged wind and meteorological parameters are then evaluated by integrating the resulting velocity profile over the swept area of the rotor. Further analyses were conducted to determine if other factors accompanying the change in turbulence level and atmospheric stability could cause or contribute to the observed sensitivity of the power curves to the turbulence and atmospheric stability as reported in the literature.

\bibliographystyle{plain}
\begin{thebibliography}{10}
\bibitem{belu_11}
{\sc J.~Sumner and C. ~Masson}. {Influence of Atmospheric Stability on Wind Turbine Power Performance Curves}. J. Sol. Energy Eng. 128(2006), pp.~531--537.

\bibitem{belu_12}
{\sc R. ~Wagner, I. ~Antoniou, S.M. ~Pedersen, M.S. ~Courtney, and H.E. ~Jorgensen}. The influence of the wind speed profile on wind turbine performance measurements. Wind Energy 12(4) (2009), pp. 348--362.
\end{thebibliography}