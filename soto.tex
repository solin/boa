
% *****************************
% *   YOUR TEXT STARTS HERE   *
% *****************************

\title{An Efficient Finite Element Scheme to Simulate Blast and Impact Coupled
  Fluid/Structure Problems}
\author{} \institute{} % Intentionally left blank
\tocauthor{\underline{Orlando Soto},~~{Joseph Baum},~~{Rainald L\"ohner}}
\maketitle
\begin{center}
{\large \underline{Orlando Soto}}\\
SAIC - 1710 Saic Dr. MSN 2-6-9 - Mclean, VA 22102, USA\\
{\tt orlando.a.soto@saic.com}\\
\vspace{4mm} % Use this space when including 3rd author
{\large Joseph Baum}\\
SAIC - 1710 Saic Dr. MSN 2-6-9 - Mclean, VA 22102, USA\\
{\tt joseph.d.baum@saic.com} \\
\vspace{4mm} % Use this space when including 3rd author
{\large Rainald L\"ohner}\\
GMU - 4400 University Drive MS 6A2 - Fairfax, VA 22030, USA\\
{\tt rlohner@gmu.edu}
\end{center}

\section*{Abstract}

In this work an efficient finite element (FE) scheme to model blast and impact
coupled fluid-solid problems is presented. Its main ingredients are: an
OSS-Q1/P0 element (tri-linear hexahedra with
constant pressure and Orthogonal Sub-grid Scale stabilization), a 
large-strain finite  element (FE) convective formulation, a phenomenological
concrete model 
to  deal with damage and fracture of concrete structures, and a general
contact algorithm  which uses bin technology to perform the node-face
searching operations in a parallel manner. All the schemes have been fully
parallelized and coupled using a loose-embedded procedure with
the CFD (computational fluid dynamics) code FEFLO.

Even though it can not be
proved that the Q1/P0 element is suitable for incompressible problems like
the pseudo-plastic flow of concrete at high pressures (the element is
not div-stable), it is utilized and works in an acceptable manner for
many real applications. Nevertheless, spurious pressure
oscillations (the chessboard mode) may
appear in several simulation time steps, which
pose a serious challenge for blast cases that require an
accurate modeling of the material damage to evaluate crack propagation and
fragment formation. Therefore, we follow the idea showed in
\cite{ref1} where linear elements have been stabilized using an OSS scheme
for flow problems: we improved the pressure
stability of the Q1/P0 
element by adding the OSS terms to the standard FE
Galerkin formulation. It can be proved that the stabilized scheme is
consistent and stable, hence, its solution converges to the solution of the
boundary value problem that is being approximated (i.e. the right physical
problem). Details of the concrete material model (K\&C), the FSI (Fluid Solid
Interaction) procedure and the contact algorithm, may be consulted in
\cite{ref3}. They will also be briefly described in the final paper. A real 3D
coupled Fluid/Structure simulation with experimental comparisons will be also
presented.

\bibliographystyle{plain}
\begin{thebibliography}{10}

\bibitem{ref1}
{\sc R.~Codina}. {On stabilized finite element methods for linear systems of
  convection-diffusion- reaction equations}. Computer Methods in Applied
Mechanics and Engineering. 188 (2000), pp.~61--82.

\bibitem{ref3}
{\sc O.~Soto, J.~Baum and R.~L\"ohner}. {An efficient fluid-solid coupled
  finite element scheme for weapon fragmentation simulations}. Engineering
Fracture Mechanics. 77 (2010), pp.~549--564.

\end{thebibliography}

% ***************************
% *   YOUR TEXT ENDS HERE   *
% ***************************
