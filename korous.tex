\title{Combined Adaptive Multimesh $hp$-FEM/$hp$-DG for Multiphysics Coupled Problems Involving Compressible Inviscid Flows}
\author{} \institute{}
\tocauthor{Lukas Korous, \underline{Milan Hanus}}

\begin{center}

\textbf{\Large Combined Adaptive Multimesh $hp$-FEM/$hp$-DG for Multiphysics Coupled Problems Involving Compressible Inviscid Flows}\\
\vspace{10mm}
{\large Lukas Korous}\\
Univerzita Karlova v Praze\\
{\tt lukas.korous@gmail.com}\\
\vspace{4mm}
{\large \underline{Milan Hanus}}\\
Zapadoceska Univerzita v Plzni\\
{\tt mhanus@kma.zcu.cz}

\end{center}

\section*{Abstract}

During the last decade, Discontinuous Galerkin (DG) methods have been studied by numerous researchers in the context of different problems ranging from linear elliptic equations to Euler equations of compressible inviscid flow, the latter e.g. in~\cite{1}. It is well known that DG methods yield larger discrete problems than standard continuous finite element methods (FEM). On the other hand, their implicit stabilization through embedded numerical fluxes makes them particularly well
suited for hyperbolic flow problems. Thus for multiphysics problems involving compressible flow we propose to combine the best of both worlds: DG is used for the flow part only while standard FEM is employed for second-order equations where it works very well. The DG/FEM combination is carried out for arbitrary-degree elements in a monolithic fashion, using a novel multimesh hp-FEM technology deployed by Dr. Solin and his collaborators~\cite{2},~\cite{3}.

\bibliographystyle{plain}
\begin{thebibliography}{10}

\bibitem{1}
{\sc V. Dolejsi, M. Feistauer, C. Schwab}. {On Some Aspects of the Discontinuous Galerkin Finite Element Method for Conservation Laws}. Math. Comput. Simul. 61 (2003), pp.~333--346.

\bibitem{2}
{\sc L. Dubcova, P. Solin, J. Cerveny, P. Kus}. {Space and Time Adaptive Two-Mesh $hp$-FEM for Transient Microwave Heating Problems}. Electromagnetics, Vol. 30, Issue 1 (2010), pp.~23--40.

\bibitem{3}
{\sc P. Solin, J. Cerveny, L. Dubcova, D. Andrs}. {Monolithic Discretization of Linear Thermoelasticity Problems via Adaptive Multimesh $hp$-FEM}. J. Comput. Appl. Math 234 (2010), pp.~2350--2357.

\end{thebibliography}