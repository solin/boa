\title{Finite element simulation of trocar insertion into human abdominal tissue}
\author{} \institute{}
\tocauthor{\underline{P.~Chavan}, Y.~W.~Seo}
\maketitle

\begin{center}
{\large \underline{Priyvardhan Chavan}, Yong Won Seo, Thenkurussi Kesavdas}\\
{\large Virtual Reality labs, State University of New York at Buffalo}\\
{\tt ppchavan@buffalo.edu, yongwons@buffalo.edu, kesh@buffalo.edu}
\end{center}

\section*{Abstract}
Trocar insertion, the first step to most micro surgery procedures is a difficult procedure to learn and practice because procedure is carried out almost entirely without any visual feedback of the organs underlying the tissue being punctured. A majority of injuries is attributed to the excessive use of force by the surgeon. Hence, the current research in Virtual reality lab is focused on simulating the trocar insertion procedure, in real time using advanced simulation techniques and develops a trocar insertion simulator for surgeon skill enhancement. In this paper, we present the results of our preliminary analysis of the problem, carried out using non linear FEM solver. The purpose of this analysis is to simulate the problem realistically with different material models of increasing complexity. The results of the analysis gives an idea about stress distribution pattern at the point of insertion, maximum stress developed in the vicinity of penetration and correlation between different trocar design factors and stress patterns in the vicinity of trocar insertion. We intend to use this information to construct a haptic simulator for surgeon skill training.

\bibliographystyle{plain}
\begin{thebibliography}{10}
\bibitem{KesavdasT2005}
{\sc Kesavdas T., Srimathveeravalli G., Arulesan V.}. {Parametric modeling and simulation of trocar insertion procedure}. Studies in Health Technology Inform, 2005, pp 119- 252-4

\bibitem{Brummer2008}
{\sc Brummer V., Carnahan H., Okrainec A., Dubrowski A.}. {Trocar insertion: the neglected task of VR simulation}. Medicine Meets Virtual Reality 16, J.D. Westwood et al. (Eds.), IOS Press, 2008

\bibitem{Okrainec2009}
{\sc Okrainec A., Farcas M., Henao O., Choy I., Green J., Fotoohi M., Leslie R., Wight D., Karam P., Gonzalez N., Apkarian J.}.
\newblock Development of a Virtual Reality Haptic Veress Needle Insertion Simulator for Surgical Skills Training
\newblock Medicine Meets Virtual Reality 17, J.D. Westwood et al. (Eds.), IOS Press, 2009

\bibitem{Shafer2006}
{\sc Shafer D., Khajanchee Y., Wong J. and Swanström L.}.
\newblock Comparison of Five Different Abdominal Access Trocar Systems: Analysis of Insertion Force, Removal Force, and Defect Size
\newblock Surgical Innovation, Volume 13 Number 3, September 2006, pp 183-189

\bibitem{SNg2003}
{\sc Ng P.S., Singh Sahota D., Yuen P.}.
\newblock Measurement of Trocar Insertion Force Using a Piezoelectric Transducer
\newblock Journal of the American Association of Gynecologic Laparoscopists, volume 10, issue 4, November 2003, pp 534-538
\end{thebibliography}