\title{Finite Element Analysis of the Effect of Multiple Abrasive Particle Impact on Erosion Behaviour of Titanium Alloy in Abrasive Water Jet Machining}
\author{} \institute{}
\tocauthor{Naresh Kumar, \underline{Mukul Shukla}}
\maketitle

\begin{center}
{\large Naresh Kumar}\\
Department of Mechanical Engineering, M.N.N.I.T.\\
{\tt nareshcadcam@gmail.com}\\
\vspace{4mm}

{\large \underline{Mukul Shukla}}\\
Mechanical Engineering Technology, University of Johannesburg\\
{\tt mshukla@uj.ac.za}
\end{center}

\section*{Abstract}
High-velocity Abrasive Water Jets (AWJ) are finding growing applications for cutting a wide range of materials such as stainless steel, concrete, titanium alloys, metal matrix composites, rocks etc. For ductile materials, the impact of garnet abrasive causes localized plastic strain at the impact surface. For brittle materials, the force of impacting particle causes localized cracking. Erosion is a rather complex phenomenon occurring in several engineering applications including abrasive waterjet machining (AWJM), slurry flow through pipes, rocket nozzles and valves. The erosion behaviour is dependent on abrasive particle angle of attack, size, shape and velocity. Titanium alloys possesses high strength to weight ratio, temperature strength, corrosion resistance and bio-compatiblity, leading to widespread use in aircraft structural members, biomedical implants etc. However, traditional machining of Ti alloys is extremely challenging and AWJM is popularly used.

In this paper AWJM modeling has been attempted using explicit finite element analysis (FEA). The main focus has been to investigate the effect of multiple particle impact on the erosion mechanism of a titanium alloy (Ti-6Al-4V), which has been unreported so far (ElTobgy et al., 2005). An elasto-plastic FE model has been developed using the LS-DYNA software. The rigid material model has been used for the impacting spherical particle and the piecewise linear elasto-plastic model for the Ti alloy workpiece. In the failure model the strain rate was accounted for using the Cowper and Symonds model which scales the yield stress by a strain rate dependent factor. The model successfully accounts for the effect of multiple (upto twenty after which appreciable changes are not evident) particle impact. The influence of abrasive particle impact angle (30, 60 and 90 degrees) and velocity (180, 200, 220 m/s) on the crater sphericity and depth of indentation produced has been investigated. The FE model was validated with experimental work of (Junkar et al., 2006) and found to be in very good agreement. The proposed FE modeling approach can largely help in limiting the experiments required to predict the erosion behaviour of Ti-6Al-4V during AWJM.

\bibliographystyle{plain}
\begin{thebibliography}{10}
\bibitem{ElTobgyElbestawi05}
{\sc M.S.~ElTobgy, E. ~Ng, and M.A.~Elbestawi}. {Finite element modeling of erosive wear}. Int. J Mach. Tools Manuf. 45, (2005), pp.~1337--1346.

\bibitem{JunkarJurisevicFajdigaGrah06}
{\sc M.~Junkar, B.~Jurisevic, M.~Fajdiga, and M.~Grah}. {Finite element analysis of single-particle impact in abrasive water jet machining}. Int. J of Impact Eng. 32, (2006), pp.~1095--1112.
\end{thebibliography}