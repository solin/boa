\title{Numerical Simulation Studies for Vertical Axis Wind Turbine Performance Prediction with Applications in Built Environment}
\author{} \institute{}
\tocauthor{\underline{R.~Belu}}
\maketitle

\begin{center}
{\large \underline{Radian Belu}}\\
Drexel University\\
{\tt rbelu@dri.edu}
\end{center}

\section*{Abstract}
Worldwide interest in renewable energy systems has increased dramatically due to environmental concerns like climate change as well as the likely future shortages in the energy supply. Wind power is a major source of sustainable energy and can be harvested using both horizontal and vertical axis wind turbines. This paper presents studies of various configurations of vertical axis wind turbines performance for built environment. Numerical simulations are presented to predict fluid flows through the various Darrrieus wind turbine configurations. Simulations of air flows through the turbine rotor were performed to analyze the performance characteristics of the devices (Agren et al. 2005). In the last four decades several aerodynamic prediction models have been formulated for wind turbine performances and characteristics. We can identified two families pf models: stream-tube and vortex (Paraschivoiu, 1988). The former needs much less computation at the expense of accuracy. The later requires more computational resources but is more accurate. This paper presents a simplified numerical technique for simulating vertical axis wind turbine flow based on the lifting line theory and a free vortex wake model including dynamic stall effects for predicting the performances of a 3-D vertical axis wind turbine. A vortex model is used in which the wake is composed of trailing stream-wise and shedding span-wise vortices whose strengths are equal to the change in the bound vortex strength as required by the Helmholz and Kelvin theorems. Using limited computer resources the stream-tube performance prediction model is capable of predicting the overall rotor power output and the distribution of aerodynamic forces along the rotor blades. Predictions by using our method are shown to compare favorably with existing experimental data and the outputs of other numerical models. The results of the numerical simulations are used in the design and optimization of wind turbines for built environment applications.     

\bibliographystyle{plain}
\begin{thebibliography}{10}
\bibitem{Paraschivoiu88}
{\sc I.~Paraschivoiu}. {Double-Multiple Streamtube Model for Studying Vertical-Axis Wind Turbines}. AIAA Journal of Propulsion and Power, 4 (1988), pp.~370--378.

\bibitem{A104}
{\sc Agren ~O. Berg O., and M. Leijon}. {A time-dependent potential flow theory for the aerodynamics of vertical axis wind turbines}. J.~Appl. Phys. 97 (2005), pp. 1--13.
\end{thebibliography}