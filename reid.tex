\title{Object-Oriented Finite Element Materials Modeling at NIST}
\author{} \institute{}
\tocauthor{A.~Reid}
\maketitle

\begin{center}
{\large Andrew Reid}
\end{center}

\section*{Abstract}
We describe an open-source materials modeling program developed at the National Institute of Standards and Technology. OOF, or ``Object Oriented Finite Elements'', is intended to provide powerful modeling and analysis capabilities to users who are experts in materials science, but not necessarily knowledgeable about computational modeling.  Such a user can start with a real or simulated image of a material microstructure, and easily create a finite-element mesh whose structure corresponds to physical features of the microstructure, then construct a model, and impose boundary conditions and perform virtual experiments to investigate the behavior of the material under load.  

The software has been developed with a focus on solid mechanics and thermal conductivity, but is easily extensible to new kinds of constitutive rules involving other fields, making it applicable to a very broad class of problems.  The current version incorporates first and second order time dependence, and sophisticated nonlinear solvers.

We are working on extending the software to fully 3D systems, as well as to various kinds of inequality-based constitutive rules, such as plasticity.