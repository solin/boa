\title{On Finite Element Application for Solution of Aeroelastic Instability of Airfoil with Consideration of Nonlinear Effects}
\author{} \institute{}
\tocauthor{P. Sv\'a\vcek}

\begin{center}

\textbf{\Large On Finite Element Application for Solution of Aeroelastic Instability of Airfoil with Consideration of Nonlinear Effects}\\
\vspace{10mm}
{\large P. Sv\' a\v cek}\\
{$^\ast$ Czech Technical University, Faculty of Mechanichal Engineering, Dep. Technical Mathematics} Karlovo nam. 13, Praha 2, 121 35,\\
{\tt Petr.Svacek@fs.cvut.cz}

\end{center}

\section*{Abstract}

This paper focuses on numerical approximation of a nonlinear problem of interactions of vicous incompressible flow over a flexibly supported airfoil with aileron.
The time-dependent computational domain is taken into account by the Arbitrary Lagrangian-Eulerian method, cf.~\cite{Sva1}. The flow is governed by the incompressible Navier-Stokes equations written in the ALE form 
\begin{equation}
\label{eq:NS}
\frac{D^{\mathcal A} {\mathbf v} }{D t} \, - \nu \triangle {\mathbf v} + (({\mathbf v}-{\mathbf w}_D)\cdot\nabla) {\mathbf v} + \nabla p = 0,  \qquad\qquad
\mbox{div}\, {\mathbf v} = 0, \qquad \mbox{ in } \Omega_t,
\end{equation}
where $\mathbf v$ is flow velocity, $p$ is the kinematic pressure and $\nu$ the kinematic viscosity. The computational domain $\Omega_t$ surrounds a flexibly supported airfoil with aileron section. The motion of the airfoil and aileron structure is described by the system of  ordinary differential equations for vertical displacement $h$, angle of rotation $\alpha$ and angle of rotation of the aileron $\beta$

\begin{equation}
\label{eq:ODE3}
\left(
\begin{array}{ccc}
m & S_\alpha & S_\beta \\
S_\alpha & I_\alpha & \widetilde{\Delta} S_\beta  + I_\beta \\
S_\beta & \widetilde{\Delta} S_\beta + I_\beta & I_\beta \\
\end{array}
\right)
\left( \begin{array}{c} \ddot{h} \\ \ddot{\alpha} \\ \ddot{\beta} \end{array} \right) + {\mathbb K} \left( \begin{array}{c} h \\ \alpha \\ \beta\end{array} \right) = \left( \begin{array}{c} -L  \\ M \\ M_\beta\end{array} \right)\,,
\end{equation}
where ${\mathbb K}=\mbox{diag}( k_{h},k_{\alpha},k_{\beta})$, $m$, $S_\alpha$, $I_\alpha$, $S_\beta$, $I_\beta$ are the structural parameters. Further, $L$, $M$, $M_\beta$ are the aerodynamical forces. In order to describe correctly the behaviour also for large displacements, the geometrical nonlinearities are included in the model. Further, the effect of nonlinear stiffness is investigated.

The numerical approximation of the presented mathematical model consists of approximations of the stucture model, the fluid model and the interface conditions. The structure model is integrated in time by Runge-Kutte 4th order method and coupled with the fluid model with the aid of strong coupling algorithm. The realization of the strongly coupled algorithm is realized by repeated solutions of both structural and flow models. For the approximation of Navier-Stokes equations the stabilized finite element method is used, cf.~\cite{Sva1}. The described method is applied on solution of several problems.

\bibliographystyle{plain}
\begin{thebibliography}{10}

\bibitem{Sva1}
{\sc P.~Sv\'{a}\v{c}ek, M.~Feistauer, J.~Hor\'{a}\v{c}ek}.
\newblock Numerical Simulation of Flow Induced Airfoil Vibrations with Large Amplitudes.
\newblock {Journal of Fluids and Structure} {23}(3) (2007), pp. 391--411.

\bibitem{Sva2}
{\sc P.~Sv\'{a}\v{c}ek}. 
On Energy Conservation for Finite Element Approximation of Flow - Induced Airfoil Vibrations.
\newblock Mathematics and Computers in Simulation 80(8) 2010, pp. 1713-1724.


\bibitem{Ricci}
{\sc S.~Ricci, A.~Scotti, J.~Cecrdle, J.~Malecek}.
Active control of three-surface aeroelastic model.
\newblock Journal of Aircraft 45(3) 2008, 1002-1013.

\end{thebibliography}