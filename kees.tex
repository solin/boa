\title{Multiscale Numerical Modeling of Levee Breach Processes}
\author{} \institute{}
\tocauthor{\underline{C. E. Kees}, M. W. Farthing, I. Akkerman, and Y. Bazilevs}

\begin{center}

\textbf{\Large Multiscale Numerical Modeling of Levee Breach Processes}\\
\vspace{10mm}

{\large \underline{C. E. Kees}, M. W. Farthing, I. Akkerman}\\
US Army ERDC, Vicksburg, MS, USA\\
{\tt christopher.e.kees@usace.army.mil, matthew.w.farthing@usace.army.mil, idoakkerman@gmail.com}\\
\vspace{4mm}
{\large Y. Bazilevs}\\
Structural Engineering, University of Caifornia, San Diego, CA, USA\\
{\tt jbazilevs@ucsd.edu}

\end{center}

\section*{Abstract}

One of the dominant failure modes of levees during flood and storm surge events is erosion-based breach formation due to high velocity flow over the back (land-side) slope.  Modeling the breaching process numerically is challenging due to both physical and geometric complexity that develops and evolves during the overtopping event. The surface water flows are aerated and sediment-laden mixtures in the supercritical and turbulent regimes. The air/water free surface may undergo perturbations on the same order as the depth or even topological change (breaking). Likewise the soil/fluid interface is characterized by evolving headcuts, which are essentially moving discontinuities in the soil surface elevation. The most widely used models of levee breaching are nevertheless based on depth-integrated
models of flow, sediment transport, and bed morphology. In this work our objective is to explore models with less restrictive modeling assumptions, which have become computationally tractable due to advances in both numerical methods and high-performance computing hardware. In particular, we present formulations of fully three-dimensional flow, transport, and morphological evolution for overtopping and breaching processes and apply recently developed finite element and level set methods to solve the governing equations for relevant test problems.