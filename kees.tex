

% *****************************
% *   YOUR TEXT STARTS HERE   *
% *****************************

\title{Multiscale Numerical Modeling of Levee Breach Processes}
\author{} \institute{} % Intentionally left blank
\tocauthor{\underline{C. E. Kees}, M. W. Farthing, I. Akkerman, and Y. Bazilevs}
\maketitle
\begin{center}
{\large \underline{C. E. Kees}}\\
US Army ERDC, Vicksburg, MS, USA\\
{\tt christopher.e.kees@usace.army.mil}\\
\vspace{4mm} % Use this space when including 3rd author
{\large M. W. Farthing}\\
US Army ERDC, Vicksburg, MS, USA\\
{\tt matthew.w.farthing@usace.army.mil}\\
\vspace{4mm} % Use this space when including 3rd author
{\large I. Akkerman}\\
US Army ERDC, Vicksburg, MS, USA\\
{\tt idoakkerman@gmail.com}\\
\vspace{4mm} % Use this space when including 3rd author
{\large Y. Bazilevs}\\
Structural Engineering, University of Caifornia, San Diego, CA, USA\\
{\tt jbazilevs@ucsd.edu}\\
\end{center}

\section*{Abstract}
One of the dominant failure modes of levees during flood and storm
surge events is erosion-based breach formation due to high velocity
flow over the back (land-side) slope.  Modeling the breaching process
numerically is challenging due to both physical and geometric
complexity that develops and evolves during the overtopping event. The
surface water flows are aerated and sediment-laden mixtures in the
supercritical and turbulent regimes. The air/water free surface may
undergo perturbations on the same order as the depth or even
topological change (breaking). Likewise the soil/fluid interface is
characterized by evolving headcuts, which are essentially moving
discontinuities in the soil surface elevation. The most widely used
models of levee breaching are nevertheless based on depth-integrated
models of flow, sediment transport, and bed morphology. In this work
our objective is to explore models with less restrictive modeling
assumptions, which have become computationally tractable due to
advances in both numerical methods and high-performance computing
hardware. In particular, we present formulations of fully
three-dimensional flow, transport, and morphological evolution for
overtopping and breaching processes and apply recently developed
finite element and level set methods to solve the governing equations
for relevant test problems.

%% Enter your abstract here. Authors of contributed lectures: Please do not 
%% exceed one page including references. References to related or 
%% competitive work are mandatory. Presenting author should be underlined. 
%% Please do not alter the internal structure of the template. Do not 
%% introduce any new definitions or commands, they cause problems during the 
%% compilation of the final Book of Abstract. The Book of Abstracts will be 
%% compiled using pdflatex. If using images, please make sure thay are
%% in PDF or PNG. Thank you!


 \bibliographystyle{plain}
 \begin{thebibliography}{10}

 \bibitem{CockburnGopalakrishnan04}
 {\sc B.~Cockburn and J.~Gopalakrishnan}. {A characterization of hybridized
   mixed methods for second order elliptic problems}. SIAM J. Numer. Anal. 42
   (2004), pp.~283--301.

 \bibitem{EwingWangYang03}
 {\sc R.~Ewing, J.~Wang, and Y.~Yang}. {A stabilized discontinuous finite
   element method for elliptic problems}. Numer. Linear Alg. Appl. 10 (2003),
   pp.~83--104.

 \bibitem{A104}
 {\sc Randolph~E. Bank, Jinchao Xu, Bin Zheng}.
 \newblock Superconvergent derivative recovery for {Lagrange} triangular
   elements of degree $p$ on unstructured grids.
 \newblock SIAM J.~Numer. Anal. 45 (2007), pp. 2032--2046. 

 \end{thebibliography}

% ***************************
% *   YOUR TEXT ENDS HERE   *
% ***************************
