\title{Study of New Leon Model for Concrete Failure}
\author{} \institute{}
\tocauthor{J.~Xu, \underline{Z.~Shen}, B.~Valentini}
\maketitle

\begin{center}
{\large  Juncai Xu, \underline {Zhenzhong Shen}}\\
College of Water Conservancy and Hydroelectric Engineering, Hohai University\\
{\tt juncaixu@yahoo.com.cn, zhzhshen@hhu.edu.cn}\\
\vspace{4mm}

{\large {Bernhard Valentini}}\\
Faculty of Civil Engineering, Innsbruck Universtiya\\
{\tt valentini@uibk.ac.at}
\end{center}

\section*{Abstract}
New Leon model (NLM) is base on a combination of the flow theory of plasticity with damage mechanics~\cite{Grassl}. But the yield function of New Leon model has two vertices at the intersection with the hydrostatic axis ~\cite{Pivonka}. Because the gradient of the yield function with respect to the stress tensor is not defined uniquely at these points. To avoid this problem, we consider the stress state is projected onto the hydrostatic axis and along the hydrostatic axis to the vertex in the second. Then this paper focuses on the comparative studies of New Leon model with Extended Leon model (ELM) in the first experiment by Hurlbut and the second experiment by Imran. Finally, the New Leon model, Extended Finite Element and gradient enhance damage model applied on plain concrete test~\cite{Feist} for simulation of concrete modeling cracking. The crack mouth opening displacement (CMOD) results were obtained from the three models. In contrast, Extended Leon model in general leads to a better agreement with the experimental results than the others.

\bibliographystyle{plain}
\begin{thebibliography}{10}
\bibitem{Grassl}
{\sc P.~Grassl, M.~Jirasek}. {Damage-plastic model for concrete failure}. International Journal of Solids and Structures. 43 (2006),pp.~7166--7196.

\bibitem{Pivonka}
{\sc P.~Pivonka}. {Comparative studies of 3D-constitutive models for concrete: application to mixed-mode fracture}. Inte.Jour.for.Num.Meth. 60 (2004), pp. 549--569.

\bibitem{Feist}
{\sc Feist}. {A Numerical Model for Cracking of Plain Concrete Based on the Strong Discontinuity Approach, }. University of Innsbruck, Ph.D. thesis,(2004), pp.~283--301.
\end{thebibliography}