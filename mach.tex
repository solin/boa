\title{Hard-coupled model of induction heating of cylindrical billets by rotation in static magnetic field}
\author{} \institute{}
\tocauthor{\underline{F.~Mach}, P.~Karban, I.~Dole\v zel}
\maketitle

\begin{center}
{\large \underline{Franti\v{s}ek Mach}, Pavel Karban}\\
Faculty of Electrical Engineering, University of West Bohemia, RICE\\
{\tt fmach@kte.zcu.cz, karban@kte.zcu.cz}\\
\vspace{4mm}

{\large Ivo Dole\v{z}el}\\
Faculty of Electrical Engineering, Czech Technical University\\
{\tt dolezel@fel.cvut.cz}
\end{center}

\section*{Abstract}
Induction heating of cylindrical billets by their rotation in static magnetic field belongs to modern and progressive technologies characterized by substantially higher efficiency in comparison with the classical processes \cite{mach1}, \cite{mach2}. The paper deals with the case when the static field is generated by a system of permanent magnets, thus without any losses in the field windings. Nowadays, this variant is explored very intensively mainly for billets of lower radii (one of somewhat simplified models was solved, for example, in \cite{mach3}).

The mathematical model of the problem consists of two nonlinear partial differential equations describing the distribution of magnetic and temperature fields (the nonlinearities rooting in the temperature dependencies of the physical parameters of individual parts of the system). Its numerical solution is carried out by a fully adaptive higher-order finite element method in the monolithic formulation using the codes Hermes2D and Agros2D \cite{mach4}. Except for the determination of the most important thermal and mechanical characteristics of heating the authors also deal with the sensitivity analysis of various nonlinearities on the results. Another comparison is performed for the results obtained on a common triangular mesh and mesh containing curvilinear elements that perfectly copies particular boundaries in the system.

This work was supported by the European Regional Development Fund and Ministry of Education, Youth and Sports of the Czech Republic under the project No. CZ.1.05/2.1.00/03.0094: Regional Innovation Center for Electrical Engineering (RICE), further by Grant project GACR P102/10/0216, and by Research Plan MSM6840770017.

\bibliographystyle{plain}
\begin{thebibliography}{10}
\bibitem{mach1}
{\sc M.~Zlobina, B.~Nacke, and A.~Nikonarov}. {Electromagnetic and thermal analysis of induction heating of billets by rotation in DC magnetic field}. Proc. UIE Krakow, Poland (2008), pp.~21--22.

\bibitem{mach2}
{\sc S.~Lupi and M. Forzan}. {A Promising high efficiency technology for the induction heating of aluminium billets}. Proc. UIE Krakow, Poland (2008), pp.~19--20.

\bibitem{mach3}
{\sc P.~Karban, F.~Mach, and I.~Dolezel}. {Higher-order finite element modeling of rotational induction heating of nonferromagnetic cylindrical billets}. Proc. HES Padua, Italy (2010), pp.~515--522.

\bibitem{mach4}
{http://hpfem.org}
\end{thebibliography}