\title{Simulation and Parameterization of Atmospheric Surface Flow with Computational Fluid Dynamics}
\author{} \institute{}
\tocauthor{\underline{J.~D.~McAlpine}, D.~Koracin}
\maketitle

\begin{center}
{\large \underline{J. D. McAlpine}}\\
Desert Research Institute, 2215 Raggio Pkwy, Reno\\
{\tt jdmac@dri.edu}\\
\vspace{4mm}

{\large Darko Koracin}\\
Desert Research Institute, Reno\\
{\tt darko.koracin@dri.edu}
\end{center}

\section*{Abstract}
The most common everyday exposure to an application of numerical modeling is the daily weather forecast. Indeed, some of the first computational modeling attempts in history were targeted towards weather prediction. The science and technique of atmospheric modeling has progressed substantially to the point that large scale features can be accurately modeled out to the limits of predictability. Much of today\textquoteright s research therefore focuses on modeling of higher frequency and smaller scale atmospheric phenomena.

At the extreme end of the scale is microscale modeling of the flow around the individual surface roughness elements in the atmospheric surface layer. This field requires direct simulation of eddies in the wake of the elements as well as thermally-driven large eddies because they greatly dominate the flow at this scale. Simulation of flow around objects can be attained using computational fluid dynamics (CFD), where the equations of motion are solved over a grid of discrete  fluid volumes, meshed around the individual roughness elements.

The difficulty in accounting for turbulent behavior requires some degree of turbulence parameterization even at this small scale. A number of methods have been developed to simulate the flow in a steady-state or unsteady fashion using various turbulence closure methods, such as k-e closure and Large Eddy Simulation. Sets of guidelines have been developed based on the work of various authors and engineering groups and recently reviewed in McAlpine and Ruby (2008).

In this work we present some of the methods that have been used to simulate the atmospheric surface layer with CFD in both neutral and unstable atmospheric conditions. Included is a review of the strengths and weaknesses of various ad-hoc and numerical procedures and guidelines established in attempts to achieve more accurate results. We also present various applications of this modeling which include environmental assessment, health and safety review, and assessment of aircraft 
operation near the surface. Recent work by the group involves CFD simulation of surface rotor wakes and dust entrainment(McAlpine, 2009).

\bibliographystyle{plain}
\begin{thebibliography}{10}
\bibitem{mcalandruby}
{\sc J.~McAlpine and M.~Ruby}. {Computational fluid dynamics of microscale meteorological flow for air quality applications}. Ch. 5C of Air Quality Modeling (P. Zannetti, Ed.), Envirocomp Inst., (2008).

\bibitem{mcalthesis}
{\sc J.~McAlpine}. {Lagrangian stochastic dispersion modeling in the atmospheric surface layer with an embedded strong flow perturbation}. Master's Thesis,  University of Nevada, Reno, (2009).
\end{thebibliography}
