\title{Crack-arresting and Strengthening Mechanism of Hybrid FRP in Strengthening of Reinforced Concrete Beams}
\author{} \institute{}
\tocauthor{X.~B.~He, Q.~G.~Yang, \underline{Q.~Shen}}
\maketitle

\begin{center}
{\large X. B. He, Q. G. Yang}\\
The MOT Key Lab. Of Bridge Structure Engineering and CQ Key Lab of Structure Engineering, Chongqing Jiaotong University\\
{\tt david.hxb@gmail.com, qgyang7053@126.com}\\
\vspace{4mm}

{\large \underline{Q. Shen}}\\
School of Civil Engineering and Architecture, Chongqing Jiaotong University\\
{\tt sq1310@126.com}
\end{center}

\section*{Abstract}
Hybrid fiber composite, which is composed of two or more complementary fibers to produce properties not presented in any individual component, possesses the capacity of crack-arresting and strengthening in cement-matrix materials. According to the principles of Fracture Mechanics, the failure process of reinforced concrete (RC) structure is exactly the emergence and propagation process of cracks. If the cracks were retarded in RC structures, the structure performance would be improved. Therefore, In this paper, Hybrid FRP (HFRP) sheets were proposed to retard crack propagation, and the crack-arresting and strengthening mechanism of HFRP sheets in the strengthening of RC beams was analyzed, which was testifiedby the finite-element-modeling (FEM) analysis and bending improvement of RC beams with externally-bonded interply Hybrid GFRP/CFRP.

\bibliographystyle{plain}
\begin{thebibliography}{10}
\bibitem{shen_11}
{\sc Z. J.~Yi, Q. G.~Yang, Z. W.~Li, C.~Zhou, and K.~P}. {A new reinforced concrete (RC) composite structure based on principles of fracture mechanics}.Damage and Fracture Mechanics VII: Computer Aided Assessment and Control (2003), pp.~455--461.

\bibitem{shen_12}
{\sc A.~Hosny, H.~Shaheen, A.~Abdelrahman, and T.~Elafandy}.{Performance of reinforced concrete beams strengthened by hybrid FRP laminates}.Cement and Concrete Composites 28 (2006), pp.~906--913.

\bibitem{shen_13}
{\sc X. B.~He, F.~Huang, and C. Y.~Zhang}. {Mechanism and experimental verification of degradation of RC beams induced by concrete/rebar bonding failure}.Key Engineering Materials 452-453 (2011), pp.~877--880.

\bibitem{shen_14}
\newblock http://www.mscsoftware.com/.
\end{thebibliography}