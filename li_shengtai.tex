\title{A Fast Mesh-Alignment Algorithm for 2D Curvilinear Grid}
\author{} \institute{}
\tocauthor{\underline{S.~Li}, J.~M.~Hyman, P.~Knupp}
\maketitle

\begin{center}
{\large \underline{Shengtai Li}, J. Mac Hyman, Mikhail Shashkov}\\
Theoretical Division, Los Alamos National Laboratory\\
{\tt sli@lanl.gov, hyman@lanl.gov}\\
\vspace{4mm}

{\large Patrick Knupp} \\
Applied Mathematics \& Applications Department, Sandia National Laboratories\\
{\tt pknupp@snl.gov}\\
\vspace{4mm}

{\large Mikhail Shashkov} \\
Computational Physics Division, Los Alamos National Laboratory\\ 
{\tt shashkov@lanl.gov}
\end{center}

\section*{Abstract}
When numerically approximating physical systems with discontinuous coefficients, often the largest numerical errors are introduced in a neighborhood of the discontinuities. These errors are often greatly reduced if the grid is aligned with the discontinuities. In predicting the extent of contamination and environment danger posed by subsurface flows from hazardous waste sites, the rate and direction of underground flows is governed by discontinuities in the geomorphology of the flow field.  Numerical approximations are more accurate when the underlying grid is  aligned with the discontinuities to minimize the heterogeneity  within a grid cell.

We will present a new numerical algorithm that aligns a curvilinear grid with internal alignment curves (IACs).  These IACs can be used to delineate internal interfaces, discontinuities in material properties, internal boundaries, or major features of a flow field. Our grid generation algorithm readjusts a predefined reference grid to create a nearby grid where the mesh cell edges are aligned with the IACs.  On an aligned grid, numerical discretizations of partial differential
equations can be formulated to satisfy the interfacial relations, such as matching fluxes across the discontinuity, to  reduce the numerical errors introduced by the discontinuity.  We present  examples to demonstrate the effectiveness of the grid-alignment algorithm for multiple imbedded interfaces.   

\bibliographystyle{plain}
\begin{thebibliography}{10}
\bibitem{CockburnGopalakrishnan04}
{\sc J.M.~Hyman, P.~Knupp, S.~Li, and M.~Shashkov}. {An algorithm to align a quadrilateral grid with internal boundary}. J. Comput. Phys. 136 (2000),  133
\end{thebibliography}