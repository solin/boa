\title{Radial Point Interpolation Method for Analysis of  Elastic Blocks Contact Problems}
\author{} \institute{}
\tocauthor{X.~Li, \underline{C.~Su}}
\maketitle

\begin{center}
{\large Xudong Li, Chao Su}\\
no1.Xikang road, Nanjing\\
{\tt perlxd@yeah.net, Csu@hhu.edu.cn}
\end{center}

\section*{Abstract}
Plenty of discontinuities surfaces and the blocks cut by them exist in the system of engineering structures. This paper apply the meshless method to analysis of discontinuous contact problems of elastic Blocks. This method adopts local radial basis point interpolation method to form the shape function and to carry out the numerical integration in the nodes on the local integral domains. This method avoids abundant mesh generation. It is conducive to numerical iterative calculation of the system of contact blocks. When the contact events are occurred between blocks, in accordance with the contact system compatibility conditions the contact objects can not invade each other and they linked each other by contact force on the contact surfaces. The Mohr-Coulomb criterion is satisfied between the surfaces of contact rock blocks. Contact situations of contact objects are determined according to the contact forces of the pairs of contact node on the surface. Then, the discrete form of global system function is inferred from energy functional by variational principle. Finallythe efficiency of the method is verified by example analysis.
Key words: meshless method; radial point interpolation method; Mohr-Coulomb criterion; contact force; energy functional

\bibliographystyle{plain}
\begin{thebibliography}{10}
\bibitem{Li01}
{\sc Jichun Li}. {Mixed methods for fourth-order elliptic and parabolic problems using radial basis functions. Advances in Computational Mathematics}. Advances in Computational Mathematics. 23 (2005), pp.~21-30.

\bibitem{KeeLiuLu03}
{\sc Bernard.~B.~T.Kee, G.~R.~Liu, C.~Lu}. {A regularized least-squares radial point collocation method (RLS-RPCM) for adaptive analysis}. Comput Mech.40 (2007), pp.~837-853.

\bibitem{GuedesAS03}
{\sc Cristina Faria Guedes,JosM.~A, Csar de S}.{A proposal to deal with contact and friction by blending meshfree and finite element methods in forming processes}.Int J Mater Form .1 (2008),pp.~177-188.

\bibitem{HeryudonoDriscoll02}
{\sc Alfa~R.~H. Heryudono and Tobin~A.~Driscoll}.{Radial Basis Function Interpolation on Irregular Domain through Conformal Transplantation}.J Sci Comput.44 (2010),pp.~286-300.

\bibitem{YanFengZhou03}
{\sc Fei Yan, Xiating Feng,Hui Zhou}.{A dual reciprocity hybrid radial boundary node method based on radial point interpolation method}.Comput Mech.45 (2010),pp.~541-552.

\bibitem{Li02}
{\sc Xin Li}.{Radial basis approximation and its application to biharmonic equation}.Adv Comput Math.32 (2010),pp.~275-302.

\bibitem{AKuttleraShillorb02}
{\sc K.~L.~Kuttlera and M.~Shillorb}.{Regularity of solutions to a dynamic frictionless contact problem with normal compliance}.Nonlinear Analysis.59 (2004),pp.~1063-1075.

\bibitem{HeintzHansbo02}
{\sc Per Heintz and Peter Hansbo}.{Stabilized Lagrange multiplier methods for bilateral elastic contact with friction}.Comput. Methods Appl. Mech. Engrg.195 (2006),pp.~4323-4333.

\bibitem{ZhangZhongWuLiao04}
{\sc H.~W.~Zhang, W.~X.~Zhong, C.~H.~Wu, A.~H.~Liao}.{Some advances and applications in quadratic programming method for numerical modeling of elastoplastic contact problems}.International Journal of Mechanical Sciences.48(2006),pp.~176-189.
\end{thebibliography}