\title{The Challenge to Provide Computational Tools for Increasingly Complex Models and Physics}
\author{} \institute{}
\tocauthor{L.~M.~Taylor}
\maketitle

\begin{center}
{\large Lee M. Taylor}\\
ANATECH Corp., San Diego, CA\\
{\tt taylor@anatech.com}
\end{center}

\section*{Abstract}
The use of computational methods in nuclear, civil and other engineering fields has become routine as the theory and algorithms have matured over the past fifty years. Nonlinear analysis including both geometric large deformations and nonlinear materials is now common. The size and complexity of the models and the necessity to perform multi-physics simulations is the big challenge for the computational mechanics community. Increasingly, the demand from the analyst community is for easy-to-use, cost-effective, time efficient analysis tools for these more complex problems. Three-dimensional analyses using the TeraGrande finite element code \cite{taylor1} will be presented to illustrate the concepts discussed in this presentation. The examples will include of aircraft impact on large pre-stressed concrete structures\cite{taylor2}, geomechanics solutions of deformations in highly faulted and stratified oil reservoirs, and blast loading on buried underground structures.

With the advent of geometry-based automatic mesh generation, engineers can now create extremely detailed finite element models that contain many parts with complex geometry. Models with millions of finite elements and hundreds of parts are now routinely generated. The ease of model generation is rapidly outpacing the ability of the computational engines to perform the analyses in a timely manner. The computational effort required for the analysis of the models is generally dominated by the CPU time spent in the equation solver and in the contact search and tracking algorithms required to determine the physical contact interactions between the parts. To reduce the turn-around time for large complex models, parallel algorithms using multiple processors are becoming common. Algorithms and the software architectures will be presented that address the issues required for successfully spanning the range from a handful of processors up to the massively parallel case of thousands of
processors in the context of both shared memory and distributed memory hardware.

%The complexity of the models is also driven by an increasing use of highly nonlinear material models that include material failure and deletion. It is no longer sufficient to simply predict the nonlinear stresses and strains in the model. Now the analyst wants to predict tearing or perforation of the parts of the model. The contact search and tracking algorithms must now determine contact between parts that have eroding surfaces. This is complicated by the requirement that these algorithms perform in a multi-processor parallel analysis. Contact algorithms will be presented that address these issues.

Geometry-based model generation using automatic mesh generation is dominated by tetrahedral finite element meshes. Tetrahedral meshes inherently have element counts much larger than comparable hexahedral meshes and exacerbate the problems of mesh size. Furthermore, mesh quality is always an issue. Some examples of tetrahedral models in the context of these issues will be presented.

\bibliographystyle{plain}
\begin{thebibliography}{1}
\bibitem{taylor1} 
\textsc{ANATECH Corp.}.
\newblock{TeraGrande User's Manual -- Version 1.0}.
\newblock{San Diego, CA, 2005.}

\bibitem{taylor2}
\textsc{Y. R. Rashid, R. J. James, et.al.}
\newblock{{Failure Analysis and Risk Evaluation of Lifeline Structures Subjected to Blast Loadings and Aircraft/Missile Impact,}}
\newblock{Proceedings of the International Workshop on Structures Response to Impact and Blast, Technion, Israel Institute of Technology, Haifa, Israel, November 15--17, 2009.}
\end{thebibliography}
