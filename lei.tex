\title{Seepage Influence due to Joint Sealing Damaged of CRFD Built on Covering Layers}
\author{Gan Lei, Shen Zhenzhong, Wang Zhi} \institute{1.State Key Laboratory of Hydrology-Water Resources and Hydraulic Engineering.2.Institude of Hydraulic Structures,Hohai University, Nanjing 210098, P. R. China} % Intentionally left blank
\tocauthor{First Author, \underline{Second Author}}

\begin{center}

\textbf{\Large Seepage Influence due to Joint Sealing Damaged of CRFD Built on Covering Layers}\\
\vspace{10mm}
{\large \underline{Gan Lei}, Shen Zhenzhong}\\
Institude of Hydraulic Structures,Hohai University\\
{\tt ganlei2015@hhu.edu.cn, shen6627@yahoo.com.cn}

\end{center}

\section*{Abstract}

The safe operation of concrete face rock fill dam what built on covering layers depends on the integrity of its water stop system. So analyzing of seepage field inside the dam before and after joint seal damage is significant, including the research of the influence on the dam seepage behavior when the water stop of peripheral joint and slab joint damaged. In this paper, the CRFD of a specific hydropower station project was used to be as a case. The numerical analysis models for simulating the different water stop damage and failure types were established. So as to compare the dam seepage before and after the joint water stop damaged and analyze the influence mechanism how the peripheral joint and slab joint damaged affect the seepage gradient , saturated surface and seepage discharge inside the dam. Base on the presenting results, the water stop area where needed to focus on can be ascertained when the dam anti-seepage system was arranged.

\bibliographystyle{plain}
\begin{thebibliography}{10}
 
\bibitem{DaweiNenghuiZhankuan 04} 
{\sc Sun Dawei, Li Nenghui, and Mi Zhankuan}. {Advance and prospect on key technology of CFRD on thick alluvium deposits}. Water Power. 08(2005), pp.~67--69.

\bibitem{LiderChernovCherdantsev 04}
{\sc A.~Lider,I.~Chernov,and Y.~Cherdantsev}. {Hydrogen Permeability of Covering Layer Generated by Electron Beam Processing}. JOURNAL OF IRON AND STEEL RESEARCH INTERNATIONAL. 17(2010), pp.~77--81.

\bibitem{Xiaoli 04}
{\sc Wen Xiaoli}. {Joint sealing design of Jiudianxia CFRD}. Gansu Water Conservancy and Hydropower Technology. 03(2008), pp.~196--197.

\bibitem{Cooke 04}
{\sc J.B.~Cooke}. {Proceedings of International Symposium on High Earth-rockfill Dams}. Chinese Soeiety for Hydro-electric Engineering.01(1993).

\bibitem{Neidert 04}
{\sc S.H.~Neidert}. {Design and construetion of the Segredo conerete-faced rockfill dam }. Water Power and Dam construetion. .06(1991).

\bibitem{BaoyuZhenzhongJian 04}
{\sc Su Baoyu, Shen Zhenzhong,and Zhao Jian}. {The cut-off negative pressure method for soling filtration problems based on the theory of variational inequalities}. JOURNAL OF HYDRAULIC ENGINEERING. 03(1996), pp.~22--29.

\bibitem{ZhenzhongChunmei 04}
{\sc Shen zhenzhong and Mao Chunmei}. {Calculation of steady seepage field and Automation draw nets}. JOURNAL OF HOHAI UNIVERSITY (NATURAL SCIENCES). 22(1994), pp.~75--77.
\bibitem{Shuwen 04}
{\sc Qi Shuwen}. {Research on Calculation Method of Seepage Discharge for Complicated 3-D Seepage Flow Field Based on FEM}. Hohai University. 05(2007).

\bibitem{ZhenzhongLiqunShuwen 04}
{\sc SHEN Zhenzhong, XU Liqun,and QI Shuwen}. {A New Interpolation Meshing Method for Calculating Section Seepage Flux}. The 10th national conference on percolation mechanics. 04(2009).

\end{thebibliography}