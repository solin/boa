\title{Numerical Simulation of Drop in a Constricted Capillary using Finite Element and Level Set Methods}
\author{} \institute{}
\tocauthor{\underline{J. Roca}, M.~S.~Carvalho}
\maketitle

\begin{center}
{\large \underline{Jose Roca}, Marcio S. Carvalho}\\
Rua Marques de Sao Vicente\\
{\tt jfrr2030@yahoo.com, msc@puc-rio.br}
\end{center}

\section*{Abstract}
Emulsion flow through a porous material can be analyzed by understanding the flow of drop suspended in a continuous phase in the pore scale. The geometry of a pore throat that connects two pore bodies is well approximated by a constricted capillary tube \cite{Carvalho01}. We study the drop dynamics in the flow through a constricted capillary tube using finite element and level set methods. The results show the variation of the pressure gradient as a function of imposed flow rate, capillary geometry, drop size, viscosity ratio of drop and continuous phase and capillary number \cite{Tsai01}. The Navier-Stokes and level set equations are used to solve the two-phase immiscible flow. The level set function in a fixed grid is used to capture the liquid-liquid interface \cite{Kim01}. Numerical results are obtained by using our FEM program developed in FORTRAN and compared to available experimental data.

\bibliographystyle{plain}
\begin{thebibliography}{10}
\bibitem{Carvalho01}
{\sc S. Cobos, M. S. Carvalho and V. Alvarado}. {Flow of oil-water emulsions through a constricted capillary}. International Journal of Multiphase Flow 35 (2009), pp.~507--515.

\bibitem{Tsai01}
{\sc T. M. Tsai and Michael J. Miksis}. {Dynamics of a drop in a constricted capillary tube}. J. Fluid Mech. vol. 274 (1994), pp.~197--217.

\bibitem{Kim01}
{\sc S. J. Kim, W. R. Hwang, K. H. Lim and H. S. Shin}. {Numerical Simulation of drop deformation in a cylindrical using moving and fixed grid methods}. International Journal of Modern Physics B. vol 22. Nos. 9,10,11 (2008), pp.~1552--1557.
\end{thebibliography}