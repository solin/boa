\title{Large-scale Density-dependent Groundwater Modelling in Agricultyrally Used Coastal Arid Regions}
\author{} \institute{}
\tocauthor{\underline{M.~Walther}, J.~Grundmann, O.~Kolditz, R.~Liedl}
\maketitle

\begin{center}
{\large \underline{Marc Walther}, Jens Grundmann, Rudolf Liedl}\\
Institute for Groundwater Management, Technische Universit\"at Dresden\\
{\tt marc.walther@tu-dresden.de}\\
\vspace{4mm}

{\large Olaf Kolditz}\\
Department of Environmental Informatics, Helmholtz-Centre for Environmental Research -- UFZ
\end{center}

\section*{Abstract}
Ensuring the sustainability of an aquifer's yield is one of the fundamental tasks for nowadays groundwater management, especially within agriculturally used regions. Due to limited water ressources in (semi-)arid regions, conflict of interests arise which need an evaluation using scenario analysis and multicriterial optimization approaches. 

Wihin the context of the government-financed research project "International Water Research Alliance Saxony" (IWAS), the groundwater quality for near-coastal, agriculturally used areas is investigated under the influence of marine saltwater intrusion. In the bounds of the study region's near-coastal areas, i.e. the Batinah plains of Northern Oman, an increasing agricultural development was observed during the recent decades. Concurrently, a steady lowering of the groundwater level was observed, which is primarily due to uncontrolled and unsupervised mining of the local aquifers. Extracted water is mainly used for the local agricultural irrigation. Intensively decreased groundwater levels, however, cause an inversion of the hydraulic gradient which is naturally aligned towards the coast. This, in turn, leads to an intrusion of marine saltwater flowing inland, endangering the productivity of farms near the coast.

Utilizing the modeling software package OpenGeoSys, which has been developed and constantly enhanced by the Department of Environmental Informatics at the Helmholtz Centre for Environmental Research Leipzig \cite{Kolditz_et_al_2008}, a three-dimensional, density-dependent model including groundwater flow and mass transport is currently being built up. The model, comprehending a part of three selected coastal wadis of interest, shall be used to investigate different management scenarios. The modelling's main focus are the optimization of well positions and pumping schemes and, in long-term view, the coupling with a surface runoff model, which is also used for the determination of the groundwater recharge due to wadi runoff downstream of retention dams. Also, the highly uncertain local hydrogeological conditions and scarcity of data are addressed by utilizing an extended inverse-distance approach for a three-dimensional interpreation of the aquifer's properties and the model's calibration. 

Based on the groundwater model, scenarios will be assessed taking various target figures into consideration, e.g. agricultural water demand, drinking water supply, beautification, tourism, industry etc. Within these scenarios, marine saltwater encroachment should be minimized or saline groundwater should even be pushed back into the coastal direction, thus stabilizing the natural equilibrium between the continental freshwater flux and seawater intrusion ensuring a long-term, stable usage of the agricultural areas.

\bibliographystyle{plain}
\begin{thebibliography}{10}
\bibitem{Kolditz_et_al_2008}
{\sc O.~Kolditz and J.-O.~Delfs, C.-M.~B\"urger, M.~Beinhorn, C.H.~Park}. {Numerical analysis of coupled hydrosystems based on an object-oriented compartment approach}. J. Hydroinformatics, 10(3): (2008), pp.~227--244.
\end{thebibliography}