\title{Large-Scale Density-Dependent Groundwater Modelling in Agriculurally Used Coastal Arid Regions}
\author{} \institute{}
\tocauthor{\underline{M.~Walther}, R.~Liedl, J.~Grundmann, O.~Kolditz}
\maketitle

\begin{center}
{\large \underline{Marc Walther}, Jens Grundmann, Rudolf Liedl}\\
Institute for Groundwater Management, Technische Universit\"at Dresden\\
{\tt marc.walther@tu-dresden.de}\\
\vspace{4mm}

{\large Jens Grundmann}\\
Institute of Hydrology, Technische Universität Dresden\\
\vspace{4mm}

{\large Olaf Kolditz}\\
Department of Environmental Informatics, Helmholtz-Centre for Environmental Research -- UFZ
\end{center}

\section*{Abstract}
Ensuring the sustainability of an aquifer's yield is one of the fundamental tasks for nowadays ground-water management, especially within agriculturally used regions. Within the context of the research project "International Water Research Alliance Saxony", the groundwater quality of near-coastal, agriculturally used areas under the influence of marine salt water intrusion is investigated. In the bounds of the study region's coastal areas, i.e. the Batinah plains of Northern Oman, an increasing agricultural development and a concurrent lowering of the groundwater level was observed during the last decades. Decreased groundwater levels cause an inversion of the hydraulic gradient which is naturally aligned towards the coast which leads to an intrusion of marine salt water endangering the productivity of farms near the coast.

Utilizing the modelling software package OpenGeoSys (Department of Environmental Informatics at the Helmholtz Centre for Environmental Research Leipzig \cite{walther1}), a three-dimensional, density-dependent model is currently being built up. The model, comprehending a part of three selected coastal wadis of interest, shall be used to investigate different management scenarios. The modelling's main focus are the optimization of pumping schemes and, in long-term view, the coupling with a surface run-off model, which is also used for the determination of the groundwater recharge due to wadi run-off downstream of retention dams. Also, the highly uncertain local hydro-geological conditions and scarcity of data are addressed by utilizing an extended inverse-distance approach for a three-dimensional interpretation of the aquifer's properties and the model's calibration.

\bibliographystyle{plain}
\begin{thebibliography}{10}
\bibitem{walther1}
{\sc OpenGeoSys}. {http://www.ufz.de/index.php?en=18345}.
\end{thebibliography}
