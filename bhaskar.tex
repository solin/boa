\title{Simulation of Fluid Flow Through an Abrasive Water Jet Nozzle using Computational Fluid Dynamics}
\author{} \institute{}
\tocauthor{A.~Bhaskar}
\maketitle

\begin{center}
{\large Abhinav Bhaskar}\\
{\tt abhinavbhaskar@gmail.com}
\end{center}

\section*{Abstract}
The present work is a compilation of efforts to gain a fundamental knowledge of the ultra-high velocity dynamic characteristics such as the velocity distribution, pressure distribution, turbulence intensity and kinetic energy of an abrasive water jet(AWJ) system. This knowledge can help enhance the AWJ technology, understanding the kerfs formation or cutting process and modelling the various cutting performance measures that are required for process control and optimization. For this purpose, Computational fluid analysis is found to be a viable approach because direct measurements of particle velocities and visualizations of particle trajectories are very difficult for the ultrahigh speed and small dimensions involved.

The development of a theoretical approach to the evaluation of turbulent flow and particle dynamic properties in the nozzle is attractive because of difficulties associated with direct measurements in nozzles of high flow speed and small dimensions. Axis symmetric simulations have been performed with the help of commercial code (Fluent 6.3), using the standard k-e turbulence model. One way coupling was considered in the simulations, which means that the effect of the presence of the dispersed solid phase in the liquid phase was not considered. The single phase coupling is considered to avoid the calculation of frictional and viscous losses in the flow, as these factors are dependent on the shape factor and sphericity of the abrasi ve particle to be used. The inside profile of the abrasive water jet nozzle was modelled by measuring the co-ordinates of the nozzle using a co-ordinate measuring machine and feeding the co-ordinates into Gambit. The profile was meshed and analysed using commercial CFD code (6.3) The results have been compared with available experimental and theoretical results published by other investigators. These results will assist in the optimal design of premixed abrasive water-jet nozzle systems and the prediction of water jet cutting performance. The turbulence characteristics can be used by the manufacturing industries for the calculation of an optimum cutting speed.