

% *****************************
% *   YOUR TEXT STARTS HERE   *
% *****************************

\title{Development of a Reactor System Analysis Code using TRILINOS}
\author{} \institute{} % Intentionally left blank
\tocauthor{Glen Hansen, \underline{Second Author}}
\maketitle
\begin{center}
{\large \underline{Glen Hansen}}\\
Idaho National Laboratory\\
{\tt Glen.Hansen@inl.gov}\\
\vspace{4mm} % Use this space when including 3rd author
{\large Vincent Mousseau}\\
Idaho National Laboratory\\
{\tt Vincent.Mousseau@inl.gov}
\end{center}

\section*{Abstract}

This presentation gives an overview of the development of RELAP6, a reactor
thermal-hydraulics code that is based on the TRILINOS multiphysics framework
from Sandia National Laboratories (\url{http://trilinos.sandia.gov}). The
software constructs a dynamic rendition of the reactor facility geometry using
models from component libraries and input files. The components interconnect to
form a simple graph, where each component supports the required degrees of
freedom (DOF) needed to support the spatial resolution needed for the analysis
of the component's thermal-hydraulic behavior. The component DOF are assembled
together with an ordering determined by the connectivity of the graph to form a
vector of nonlinear algebraic equations.  The component graph can construct an
initial condition vector given the initial state of the components, or can
create a residual vector that is valid for any time state of the system. These initial
conditions and system residual is used by TRILINOS' NOX nonlinear solver, various linear solvers
and preconditioner packages, to obtain the state of the system each time step. Typically, this
provides an implicit Jacobian-free Newton Krylov solution method that provides for stable, parallel
system simulation given the user's choice of time step.

The software system leverages the CMAKE build system within TRILINOS and the CTEST 
testing system for regression testing. Details of the software architecture, capabilities, and
design strategy will be discussed in detail.

\bibliographystyle{plain}
%\bibliography{mrabbrev,localabbrev,external,mmg_journals,reports}
\begin{thebibliography}{1}

\bibitem{gaston09a}
D.~Gaston, G.~Hansen, S.~Kadioglu, D.~Knoll, C.~Newman, H.~Park, C.~Permann,
  and W.~Taitano.
\newblock Parallel multiphysics algorithms and software for computational
  nuclear engineering.
\newblock {\em Journal of Physics: Conference Series}, 180(1):012012, 2009.

\bibitem{pope:09}
M.~A. Pope and V.~A. Mousseau.
\newblock Accuracy and efficiency of a coupled neutronics and thermal
  hydraulics model.
\newblock {\em Nuclear Engineering and Technology}, 41(7):885--892, 2009.

\end{thebibliography}


% ***************************
% *   YOUR TEXT ENDS HERE   *
% ***************************
