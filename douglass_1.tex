\title{A Python Implementation of the Wilson-Fowler Spline for Open and Closed Curves}
\author{} \institute{}
\tocauthor{\underline{R.~W.~Douglass}, L.~M.~Lang}
\maketitle

\begin{center}
{\large \underline{Rod W. Douglass}, Laura M. Lang}\\
XCP-1, MS T085, Los Alamos National Laboratory\\
{\tt rwd@lanl.gov, llang@lanl.gov}
\end{center}

\section*{Abstract\footnote{The title and abstract have been approved for unrestricted release as Los Alamos 
National Laboratory report LA-UR 11-00577.}}
We present a Python-class implementation of the Wilson-Fowler spline algorithm as outlined in \cite{Wilson66} and as modified by Melvin \cite{Melvin82}.  That is, consider a set of $N$ discrete points in two-dimensional space, $\{P_i = (x_i, y_i), i = 1,\ldots,N\}$ so that if $P_1 = P_{N}$, the resulting curve is closed.  Define a set of continuous cubic Hermitian curves, $\Omega_s(u,v)$,  for local (segment) coordinates $(u,v)$ defined on each of the $s = 1,\ldots,N-1$ segments spanning $P_s$ to $P_{s+1}$. The curve, $\Omega$, is a Wilson-Fowler spline if \cite{Melvin82}: for each $s = 1, \ldots, N-1$, $\Omega$, restricted to the segment from $P_s$ to $P_{s+1}$, is a member of $\Omega_s$, $\Omega$ has a continuous slope and curvature, and $\Omega$ has tangent vectors, $\vec{T}_1$ at $P_1$ and $\vec{T}_{N}$ at $P_{N}$, if $P$ forms an open curve.  For closed curves, the tangents are found by applying the algorithm to the first/last point, forming a "cyclic" non-linear algebraic system.  The non-linear algebra problem for the tangents at each node is solved using Newton's method with a line-search update (based on pg. 383, ff. \cite{Press92}).

Once the spline is computed, methods were written to perform operations on the spline.  These include ray-spline intersection (based on \cite{Williams03}), insertion of $m$-points between original points, and point re-distribution methods.  Point insertion and re-distribution may use either equal arc-length, equal angles, or equal Euclidean length between points criteria.

Results demonstrate the efficacy and accuracy of the spline for a variety of test problems.

\bibliographystyle{plain}
\begin{thebibliography}{10}
\bibitem{Wilson66}
{\sc A. H..~Fowler and C. W.~Wilson}. {Cubic spline, A curve fitting routine}. Report No. Y-1400 (Revision 1), Oak Ridge National Laboratory (1966).

\bibitem{Melvin82}
{\sc W. R..~Melvin}. {Error analysis and uniqueness properties of the Wilson-Fowler spline}. Los Alamos National Laboratory Report LA-9178, (1982).

\bibitem{Williams03}
{\sc Amy~Williams, Steve~Barrus, R.~Keith~Morley, and Peter~Shirley}. {An efficient and robust ray-box intersection algorithm}. J. Graphics Tools, 10 (2003). pg. 54 ff.

\bibitem{Press92}
{\sc W.H~Press, S.A.~Teukolsky, W.T.~Vettering, and B.P.~Flannery}. {Numerical Recipes in C: The Art of Scientific Computing}. Second Edition, Cambridge University Press, New York (1992).
\end{thebibliography}