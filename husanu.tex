\title{Vaporization and Combustion of Droplet Clusters in Zero Gravity Environment}
\author{} \institute{}
\tocauthor{\underline{I.~C.~Husanu}, R.~Belu}
\maketitle

\begin{center}
{\large \underline{Irina Ciobanescu Husanu}, Radian Belu}\\
Drexel University\\
{\tt inc22@drexel.edu, rbelu@dri.edu}
\end{center}

\section*{Abstract}
Isolated drop and drop array studies are common methods to isolate and investigate the effects of many of the complexities that enter into the drop combustion process. While the combustion of small drops (1 - 100 micro m diameters) is not greatly affected by buoyancy, these drops are difficult to observe. Microgravity environments are required to allow larger drops to be studied while minimizing or eliminating the confounding effects of buoyancy[1],[2]. Even with the large number of isolated droplet, droplet array, and spray studies that have been conducted in recent years, the extrapolation of the results from droplet array studies to spray flames is difficult since even the simplest spray systems introduce complexities of multi-disperse drop sizes and drop-drop interactions, coupled with more complicated fluid dynamics. Fiber-supported droplet array studies attempt to bridge the gap between individual group combustion and spray flames yet they, too, are limited because the fiber interactions introduce additional unknowns into the problem. The objective of this paper is to study the vaporization of well-characterized clusters in microgravity environment using direct numerical simulation. The research investigates droplet interactions during vaporization of clusters of droplets of different sizes and asymmetric three-dimensional configurations in zero gravity environments for low relative Reynolds numbers. The numerical simulation accounts for variable thermo-physical properties, includes the gas-phase radiative transfer for finite rate reaction and simulates the variation of the fuel mass flow rate with the radius of the droplets in the cluster. Mass burning rates are calculated for each droplet in an array and compared to mass burning rate of similar single droplet, the ratio of these two being a correction factor that depends on droplet diameters and droplets interspacing in cluster. Preliminary data obtained with proposed method provided results consistent with and in qualitative agreement with single droplet combustion theories[1]. These simulations may support future possible microgravity experiments. Using this system, dilute and dense clusters can be created and stabilized before combustion is begun allowing the spectrum of droplet interactions during combustion to be observed and quantified.

\bibliographystyle{plain}
\begin{thebibliography}{10}
\bibitem{AnnamalaiRyan93}
{\sc K.~Annamalai and W.~Ryan}. {Evaporation of Arrays of Drops Using the Point Source Method}. ASME Publications, Emerging Energy Technology, PD vol. 50 (1993)

\bibitem{MoriueaMikamiKojimaEigenbrod05}
{\sc O.~Moriuea, M.~Mikami, N.~Kojima, and C.~Eigenbrod}. {Numerical simulations of the ignition of n-heptane droplets in the transition diameter range from heterogeneous to homogeneous ignition}. Proc. of Comb. Inst. 30 (2005)
\end{thebibliography}