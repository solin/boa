\title{Using $hp$-FEM to model actuation of IPMC materials}
\author{} \institute{}
\tocauthor{\underline{D.~Pugal}, P.~Solin, K.~J.~Kim, A.~Aabloo}
\maketitle

\begin{center}
{\large \underline{David Pugal}, Kwang J. Kim, Alvo Aabloo}\\
Mechanical Engineering Department, University of Nevada\\
Institute of Technology, Tartu University\\
{\tt david.pugal@gmail.com, kwangkim@unr.edu, alvo.aabloo@ut.ee}\\
\vspace{4mm}

{\large Pavel Solin}\\
Department of Mathematics and Statistics, University of Nevada\\
Institute of Thermomechanics, v.v.i., Academy of Sciences of the Czech Republic\\
{\tt solin@unr.edu}
\end{center}

\section*{Abstract}
The system of Poisson and Nernst-Planck (PNP) equations is used to describe the charge transport in ionic polymer-metal composite~\cite{shahinpoor2001} (IPMC) materials. Deformation of the material can be described by coupling the calculated charge density to the local body forces and solving Navier's equation. The charge density occurs in a very narrow region near the boundaries, characterized by the Debye length that is in the range of micrometer or less~\cite{bazant2004diffuse}. The required computing power for a full scale finite element (FE) model is, especially in 3D, rather significant.
  
First, we briefly demonstrate how using the multi-meshing and time dependent adaptivity when solving the PNP system with Hermes~\cite{Hermes-project} result in the most optimal mesh in terms of computation time and problem size for each time step while maintaining a pre-set error of solution. Secondly, we show the multi-mesh and $hp$-adaptivity performance when calculating the actuation of IPMC --- that is a solution of the nonlinear system of four equations. Also, modeling IPMC actuation in 3D is discussed while preliminary results are demonstrated.

\bibliographystyle{plain}
\begin{thebibliography}{10}
\bibitem{shahinpoor2001}
\textsc{M.~Shahinpoor and K.~J. Kim.}
\newblock Ionic polymer-metal composites: I. fundamentals.
\newblock {\em Smart Materials \& Structures}, 10(4):819--833, 2001.

\bibitem{bazant2004diffuse}
\textsc{M.Z. Bazant, K.~Thornton, and A.~Ajdari.}
\newblock {Diffuse-charge dynamics in electrochemical systems}.
\newblock {\em Physical Review E}, 70(2):21506, 2004.

\bibitem{Hermes-project}
\textsc{P.~Solin and et~al.}
\newblock {\em Hermes - Higher-Order Modular Finite Element System (User's Guide)}.
\end{thebibliography}