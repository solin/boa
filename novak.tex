
% *****************************
% *   YOUR TEXT STARTS HERE   *
% *****************************

\title{Compression of Large Microstructural Data Sets via Aperiodic Tiling}
\author{} \institute{} % Intentionally left blank
\tocauthor{\underline{Jan Nov\'{a}k}, Ani\v{c}ka Ku\v{c}erov\'{a}} \maketitle
\begin{center}
{\large \underline{Jan Nov\'{a}k}}\\
CTU in Prague, Faculty of Civil Engineering, Centre of Integrated Design of Advanced Structures, Th\'akurova 7, Prague, Czech Republic\\
{\tt novakj@cml.fsv.cvut.cz}\\
\vspace{4mm} % Use this space when including 3rd author
{\large Ani\v{c}ka Ku\v{c}erov\'{a}}\\
CTU in Prague, Faculty of Civil Engineering, Department of Mechanics, Th\'akurova 7, Prague, Czech Republic\\
{\tt anicka@cml.fsv.cvut.cz}\\
\end{center}

\renewcommand{\note}[1]{{\noindent\color{red}[#1]}}

\section*{Abstract}

%\note{what's the main topic}\\
The main objective of this paper is drawn toward the compression of large datasets determining the microstructure of macroscopic bodies or, vice versa, a qualitatively new technique of filling the infinite Euclidean space by an non-periodic microstructure from only few representative volume elements.
%\note{what's the method based on}\\
The proposed method is based on the concept of aperiodic tiling making use the lowest aperiodic set of Wang tiles discovered so far~\cite{Culik96}. Exploiting this approach, the entire information about the infinitely large 2D microstructure is reduced to only 13 squared tiles along with five different edges. It is worthwhile to note that the essential number of tiles to compose a three dimensional domain can be even smaller.

%\note{how and motivation}\\
Our goal is to investigate the effectiveness of optimization
algorithms in forcing the pre-determined properties of required
microstructure~\cite{Leps05}. Basically, it wrestles several
constrains of two kinds. The first kind of issues, represented by
the physical nature of simulated material, consists of forcing the
prescribed volume fraction of each material's phase and
characteristic length of the microstructure. Yet, the second
concern is more artificial, as it includes the problem of
interaction between contiguous tile edges and required overall
materials behaviour (isotropy/anisotropy/orthotropy). The
performance of proposed methodology is demonstrated on a simple
planar benchmark of generating the microstructure of two-phase
inclusion-like composite material. The method shows the potential
to deal with one of the fundamental demerits of the recently
emerging hybrid concept to analysis of microstructured materials
across all scales of interest without the need of
averaging~\cite{Novak10}.

\bibliographystyle{plain}
\begin{thebibliography}{10}

\bibitem{Culik96}
{\sc K.~Culik}.
{An aperiodic set of 13 Wang tiles}. Discrete Mathematics 160 (1996), pp.~245--251.

\bibitem{Leps05}
{\sc M.~Lep\v{s}}. {Single and Multi-Objective Optimization in Civil Engineering with Applications}, PhD. thesis, CTU in Prague, (2005).

\bibitem{Novak10}
{\sc J.~Nov\'{a}k, \L.~Kaczmarczyk, P.~Grassl, C.~J.~Pearce}. {Hybrid Analysis of Fibre Reinforced Polypropylene from Micro to Macro Scale}. Proceedings of the 10th International Conference on Computational Structures Technology, (2010).

\end{thebibliography}

% ***************************
% *   YOUR TEXT ENDS HERE   *
% ***************************
