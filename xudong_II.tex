
% *****************************
% *   YOUR TEXT STARTS HERE   *
% *****************************

\title{Title of Your Talk (Note Capitalization)}
\author{} \institute{} % Intentionally left blank
\tocauthor{First Author, \underline{Second Author}}
\maketitle
\begin{center}
{\large First Author}\\
Address of First Author\\
{\tt email@of.first.author}\\
\vspace{4mm} % Use this space when including 3rd author
{\large \underline{Second Author}}\\
Address of Second Author\\
{\tt email@of.second.author}
\end{center}

\section*{Abstract}

Plenty of discontinuities surfaces and the blocks cut by them exist in the system of engineering structures. This paper apply the meshless method to analysis of discontinuous contact problems of elastic Blocks. This method adopts local radial basis point interpolation method to form the shape function and to carry out the numerical integration in the nodes on the local integral domains. This method avoids abundant mesh generation. It is conducive to numerical iterative calculation of the system of contact blocks. When the contact events are occurred between blocks, in accordance with the contact system compatibility conditions the contact objects can not invade each other and they linked each other by contact force on the contact surfaces. The Mohr-Coulomb criterion is satisfied between the surfaces of contact rock blocks. Contact situations of contact objects are determined according to the contact forces of the pairs of contact node on the surface. Then, the discrete form of global system function is inferred from energy functional by variational principle. Finally��the efficiency of the method is verified by example analysis.
Key words: meshless method; radial point interpolation method; Mohr-Coulomb criterion; contact force; energy functional
Reference:
1.W.Chen,M.TANKA(2002)A Meshless, Integration-Free, and Boundary-Only RBF Technique. Computers and Mathematics with Applications 43: 379-391.
2.Jichun Li(2005) Mixed methods for fourth-order elliptic and parabolic problems using radial basis functions. Advances in Computational Mathematics 23: 21-30.
3.Bernard B. T. Kee ,G. R. Liu ,C. Lu(2007) A regularized least-squares radial point collocation method (RLS-RPCM) for adaptive analysis.Comput Mech 40:837-853.
4.Mehdi Dehghan , Ali Shokri(2007) A numerical method for KdV equation using collocation and radial basis functions. Nonlinear Dyn 50:111-120.
5.Cristina Faria Guedes , Jos�� M. A, C��sar de S��(2008) A proposal to deal with contact and friction by blending meshfree and finite element methods in forming processes. Int J Mater Form 1:177-188.
6.Alfa R.H. Heryudono ,Tobin A. Driscoll(2010) Radial Basis Function Interpolation on Irregular Domain through Conformal Transplantation. J Sci Comput 44: 286-300.
7.Fei Yan �� Xiating Feng �� Hui Zhou (2010) A dual reciprocity hybrid radial boundary node method based on radial point interpolation method. Comput Mech 45:541-552.
8.Xin Li(2010) Radial basis approximation and its application to biharmonic equation.Adv Comput Math 32:275-302.
9.Michael Scheuerer(2011) An alternative procedure for selecting a good value for the parameter c in RBF-interpolation, Adv Comput Math 34:105-126.
10.P.Alart(1992)A mixed formulation for frictional contact problems prone to Newton like solution methods. Computer methods in applied mechanics and Engneerning 92:353-375.
11.K.L. Kuttlera, M. Shillorb(2004)Regularity of solutions to a dynamic frictionless contact problem with normal compliance. Nonlinear Analysis 59: 1063 - 1075.
12.Per Heintz ,Peter Hansbo(2006)Stabilized Lagrange multiplier methods for bilateral elastic contact with friction. Comput. Methods Appl. Mech. Engrg. 195:4323-4333.
13.H.W. Zhang, W.X. Zhong, C.H. Wu, A.H. Liao(2006) Some advances and applications in quadratic programming method for numerical modeling of elastoplastic contact problems. International Journal of Mechanical Sciences 48:176-189.
14.A.Konyukhov, K. Schweizerhof(2006)A special focus on 2D formulations for contact problems using a covariant description. Int. J. Numer. Meth. Engng 66:1432-1465.
15.Markus Gitterle, Alexander Popp, Michael W. Gee, Wolfgang A. Wall(2010) Finite deformation frictional mortar contact using a semi-smooth Newton method with consistent linearization. Int. J. Numer. Meth. Engng 84:543-571.
16.Young Ju Ahn (2010) Discontinuity of quasi-static solution in the two-node Coulomb frictional system. International Journal of Solids and Structures 47:2866-2871.



\bibliographystyle{plain}
\begin{thebibliography}{10}

\bibitem{CockburnGopalakrishnan04}
{\sc B.~Cockburn and J.~Gopalakrishnan}. {A characterization of hybridized
  mixed methods for second order elliptic problems}. SIAM J. Numer. Anal. 42
  (2004), pp.~283--301.

\bibitem{EwingWangYang03}
{\sc R.~Ewing, J.~Wang, and Y.~Yang}. {A stabilized discontinuous finite
  element method for elliptic problems}. Numer. Linear Alg. Appl. 10 (2003),
  pp.~83--104.

\bibitem{A104}
{\sc Randolph~E. Bank, Jinchao Xu, Bin Zheng}.
\newblock Superconvergent derivative recovery for {Lagrange} triangular
  elements of degree $p$ on unstructured grids.
\newblock SIAM J.~Numer. Anal. 45 (2007), pp. 2032--2046.

\end{thebibliography}

% ***************************
% *   YOUR TEXT ENDS HERE   *
% ***************************
