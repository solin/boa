\title{Study on the calculation method of contact between foundation and shiplock on soft foundation}
\author{} \institute{}
\tocauthor{Chao Xu, Chao Su,\underline{Lin Su}}

\begin{center}

\textbf{\Large Study on the calculation method of contact between foundation and shiplock on soft foundation}\\
\vspace{10mm}
{\large Chao Xu, Chao Su, \underline{Lin Su}}\\
College of Water Conservancy and Hydropower Engineering,HoHai University\\
{\tt xuchao571@163.com, csu\_hhu@163.com, sulin1987@tamu.edu}

\end{center}

\section*{Abstract}

To construct reinforced concrete ship lock on a soft foundation, the simulation of contact between concrete block and foundation soil mass is one of key problems in structural simulation analysis. With semi-analytical method, foundation is calculated with analytical formula of semi-infinite elastic foundation subsidence, upper concrete structure is discretized with eight-node three-dimensional isoparametric element. According to the basic idea of displacement method and force method from classical mechanics, governing equations which take interfacial reactions and structural global movements as variables can be established. Based on the iterative calculation of governing equations, contact problem between upper concrete structure and foundation soil mass can be solved and deformation and stress of structure and foundation are computed. It has been applied to simulation of jiangsu lusi shiplock successfully.

\bibliographystyle{plain}
\begin{thebibliography}{10}

\bibitem{A01}
{\sc Chao Su, Hongdao Jiang, Enhui Tan}. {Simulating Computation Method and Its Application to Structures on Soft Foundations}. Journal of HoHai University. 28 (2000), pp.~23--28.

\bibitem{A02}
{\sc Chao Su, Hongdao Jiang, Enhui Tan}. {An inverse analysis method for foundation parameters and its application based on viscoelastic foundation beam computation}. Chinese Journal of Geotechnical Engineering. 22 (2000), pp.~186--189.

\bibitem{A03}
{\sc Chao Su, Xianmei Wang, Jianzhong Cao, Fei Lu, Jianxin Lu, Zhaoming Ding}. {Simulation method for shiplock structure}. Port \& Waterway Engineering. 9 (2010), pp.~97--104.

\bibitem{A04}
{\sc Zuoxin Fu}. {Some Problems in the design and construction of Large-size Lock Floor}. Port \& Waterway Engineering. 1 (1993), pp.~37--42.

\end{thebibliography}