\title{Parallel FEM Simulation of Electromechanics of the Heart}
\author{} \institute{}
\tocauthor{H.~Xia, K.~Wong, X.~Zhao}
\maketitle

\begin{center}
{\large Henian Xia, Kwai Wong, and Xiaopeng Zhao}\\
Department of Mechanical, Aerospace, and Biomedical Engineering\\
National Institute for Computational Sciences, University of Tennessee\\
{\tt xzhao9@utk.edu}
\end{center}

\section*{Abstract}
In this work, we develop a FEM model to investigate electromechanics problems in the heart. The model integrates properties of cardiac electrophysiology, electro-mechanics, and mechano-electrical feedback. In order to account for large deformations of heart muscle, the updated Lagrangian approach is adopted. The parallel numerical model uses Trilinos and is implemented on the supercomputer, Kraken, at the National Institute for Computational Sciences (NICS).
Extensive numerical simulations are carried out on a dog ventricle to investigate the interaction of electrical and mechanical functions in the heart and their influences to cardiac arrhythmias.
Fatal arrhythmias may cause sudden cardiac arrest (SCA), a leading cause of death in the industrialized world, claiming over 350,000 Americans each year.