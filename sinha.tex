

% *****************************
% *   YOUR TEXT STARTS HERE   *
% *****************************

\title{Analysis of Pharmaceutical Tablet Compaction using Finite Element Method}
\author{} \institute{} % Intentionally left blank
\tocauthor{Tuhin Sinha, \underline{Carl Wassgren}}
\maketitle
\begin{center}
{\large Tuhin Sinha}\\
School of Mechanical Engineering, Purdue University, West Lafayette, IN 47907\\
{\tt sinha0@purdue.edu}\\
\vspace{4mm} % Use this space when including 3rd author
{\large \underline{Carl Wassgren}}\\
School of Mechanical Engineering, Purdue University, West Lafayette, IN 47907\\
{\tt wassgren@purdue.edu}
\end{center}

\section*{Abstract}

Pharmaceutical tablets are the most common form of drug dosage and have many advantages over other drug delivery forms such as ease of handling, storage, distribution, etc. thus, making tabletting as one of the most significant unit operations in the pharmaceutical industry. However, there are many factors that lead to poorly compacted tablets prone to attrition and breakage during various post-compaction processes (e.g., tablet coating) and other handling processes. Our study involves the application of finite element methods to investigate the details of tablet compaction process. The pharmaceutical powder is treated as a continuum with its elastic-plastic constitutive behavior being described by the Drucker-Prager Cap (DPC) model. The model captures the shearing and densification of the powder during compaction, predicting the localized stress and relative density distributions, thus providing a detailed insight of the mechanics of powder compaction. The DPC model was implemented for cylindrical shaped tablet geometry through commercial FEA software, Abaqus for Micro-crystalline cellulose (MCC Ph101), with the material parameters being taken from the relevant literature. The elastic-plastic material parameters of the constitutive model are strongly dependent on the relative density (porosity) of the powder and evolve gradually with it during the tablet compaction process. In the current study, we have developed a modified material model in which the DPC parameters are dependent on the real-time relative density of the powder during compaction and hence provide more accurate macroscopic property distributions within the tablet. The results show remarkable differences in the final stress and relative density distributions and provide the detailed physics of powder flow during compaction as compared to the previous works that adopted constant material parameters in the simulations.
 

% ***************************
% *   YOUR TEXT ENDS HERE   *
% ***************************
