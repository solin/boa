
% *****************************
% *   YOUR TEXT STARTS HERE   *
% *****************************

\title{Numerical Study on Stability of Surrounding Rock with Parallel Weak Interlayer}
\author{} \institute{} % Intentionally left blank
\tocauthor{First Author, \underline{Second Author}}
\maketitle
\begin{center}
{\large Yusong Jiang}\\
College of Water Conservancy and Hydropower, Hohai University\\
{\tt jyscn@126.com}\\
\vspace{4mm} % Use this space when including 3rd author
{\large \underline{Chao Su}}\\
College of Water Conservancy and Hydropower, Hohai University\\
{\tt csu@hhu.edu.cn}
\end{center}

\section*{Abstract}

Losing stability of surrounding rock is closely related to discontinuities during underground excavation. Aiming at the defects of numerical simulation methods for stability of surrounding rock with various discontinuity surfaces, the accuracy of FEM for contact problems to simulate discontinuities are pointed out. With FEM code for solving multibody contact problem developed by the authors, this paper mainly study the effect of different dip angles and geo-stress on stability of surrounding rock with parallel weak interlayer. Numerical analysis indicates that both the dip angles and geo-stress have a strong impact on distortion and stress redistribution of surrounding rock. The research results may provide theoretical reference for engineering practice.

\bibliographystyle{plain}
\begin{thebibliography}{10}

\bibitem{Zhou01}
{\sc Weiyuan Zhou}. {Higher rock mechanics}. China WaterPower Press.
  (1993)

\bibitem{ZhuRuanLi02}
{\sc Weishen Zhu,Yansheng Ruan, and Xiaojing Li}. {Abnormal stress distribution adjacent to a fault and its influence on stability of tunnel}. Chinese Journal of Underground Space and Engineering. 4 (2008),
  pp.~685--689.

\bibitem{HuangZhang03}
{\sc Da Huang,Runqiu Huang,Yongxing Zhang}. {Analysis on influence of fault location and strength on deformation and stress distribution of surrounding rocks of large underground openings}.Journal of Civil,Architectural \& Environmental Engineering. 31 (2009),
  pp.~68--73.

\bibitem{ZhangLiSwoboda04}
{\sc Zhiqiang Zhang,Ning Li,G.~Swoboda}. {Influence of weak interbed distribution on stability of underground openings}.  Chinese Journal of Rock Mechanics and Engineering. 24 (2005),
  pp.~3252--3257.

\bibitem{JeonKimSeo05}
{\sc S.~Jeon, J.~Kim, and Y.~Seo}. {Effect of a fault and weak plane on the stability of a tunnel in rock��a scaled model test and numerical analysis}. International Journal of Rock Mechanics and Mining Sciences. 41 (2004),
  pp.~658--663.

 
\bibitem{HAzzam06}
{\sc Y.~H. Hao,R.~Azzam}.
\newblock The plastic zones and displacements around underground openings in rock masses containing a fault.
\newblock Tunnelling and Underground Space Technology. 46 (2005), pp. 97--106.

\bibitem{JiangSu07}
{\sc Yusong Jiang.Chao Su}.
\newblock Mixed finite element method for contact problems of multibody.
\newblock Earth and Space 2010: Engineering, Science, Construction,
and Operations in Challenging Environments. (2010), pp. 606--620.


\end{thebibliography}

% ***************************
% *   YOUR TEXT ENDS HERE   *
% ***************************
