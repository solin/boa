\title{Reduced-Order Modeling of Two-Phase Flow Hydraulics using Orthonormal Wavelets}
\author{} \institute{}
\tocauthor{M.~Hernandez}
\maketitle

\begin{center}
{\large \underline{Miguel Hernandez IV}, Leticia Velazquez}\\
Department of Mathematical Sciences, The University of Texas at El Paso\\
{\tt \{miguelher,leti\}@utep.edu}\\
\end{center}

\section*{Abstract}
We present the numerical performance of the solution to nonlinear least-squares problems with reduced-order models using orthonormal wavelets.  The reduced-order wavelet modeling scheme is tested on a two-phase flow hydraulics problem using the Proteus Adaptive Hydraulics Modeling System (previously known as the Python Adaptive Hydraulics Modeling System [PyADH]) developed by the U.S. Army Corps of Engineers, Engineer Research and Development Center, Coastal and Hydraulics Laboratory.  Using Proteus, the goal is to estimate the unknown permeability field such that the difference in pressure between the model and the observed behavior of the system is minimized.

Solutions to the nonlinear least-squares problems with reduced-order models will be obtained using the Simultaneous Perturbation Stochastic Approximation (SPSA) algorithm, a stochastic steepest descent direction algorithm.  In this work, the wavelets used for model order reduction include the Daubechies, Coiflet, and Symlet families of wavelets.  In addition to a comparison of the various orthonormal wavelet families, an analysis of the performance of each at different levels of decomposition will also be discussed.

\bibliographystyle{plain}
\begin{thebibliography}{10}
\bibitem{MHWaveletModel}
{\sc M.~Hernandez~IV and L.~Velazquez and M.~Argaez}. {A Comparison of Wavelet-Based Schemes for Parameter Estimation}. Proceedings of the IEEE 2010 Users Group Conference. (2010)

\bibitem{NumTestParam}
{\sc L.~Velazquez and M.~Argaez and C.~Quintero}. {Numerical Testing of Parameterization Schemes for Solving Parameter Estimation Problems.} Proceedings of the 26th Army Science Conference (2008)
\end{thebibliography}